%D�finir le format du document: papier, taille de police, type de document, etc.
\documentclass[a4paper, 11pt]{article}

%%%%%%%%% Packages externes utilis�s %%%%%%%%%%%%%%%%%%%
\usepackage[french]{babel}
\usepackage[latin1]{inputenc}
\usepackage[T1]{fontenc}
\usepackage{verbatim}
\usepackage{graphicx}
\usepackage{epstopdf}
\usepackage{macro}
\usepackage{algorithm}
\usepackage{algorithmic}


%La mise en page du rapport, NE PAS MODIFIER.
\usepackage{geometry}
 \geometry{
 a4paper,
 left=20mm,
 right=20mm,
 top=20mm,
 bottom=20mm,
 }

%%%%%%%%% Le corps du document entre begin et end %%%%%%%%%%%%%%%%%%%
\begin{document}

%Page de garde
%%%%%%%%%%%%%%% Page de garde %%%%%%%%%%%%%%%%%%%

\begin{titlepage}{
    \begin{center}
				\noreffig{images/UCP_logo.jpg}{5cm}{3.5cm}
        \skip {0}
        {\Large \textbf {Universit� de Cergy-Pontoise}} \\
        \vspace* {10mm}
        {\Large \textbf {Rapport de Projet}} \\
        \vspace* {10mm}
        pour l'Unit� d'Enseignement "G�nie Logiciel et Programmation" \\
        \textbf {Licence d'Informatique 2e Ann�e} \\
        \vspace* {10mm}

	sur le sujet \\
        \vspace* {10mm}
	{\Huge \textbf{Histoire}} \\
        \vspace* {10mm}
 	r�dig� par \\
        \vspace* {10mm}
        {\Large \textbf {Matteo STAIANO, Mathieu HANNOUN}} \\
				\vspace* {10mm}
				\noreffig{images/histoire.jpg}{13cm}{7cm} \\
        \date Mai 2017
        \vspace* {10mm}
	\end{center}
}
\end{titlepage}


%G�n�ration automatique de la table des mati�res, de la liste des figures et de la liste des tableaux
\tableofcontents
\listoffigures
\listoftables

%Une section "remerciements" pourrait �tre int�ressante. C'est une section non num�rot� (avec un * )
\section*{Remerciements}
Les auteurs du projet voudraient remercier...
\newpage
\section{Pr�sentation du projet}
\label{sec:presentation}

\subsection{Contexte}
Le module de G�nie Logiciel et Programmation de L2 nous demandant la r�alisation d'un projet en Java et souhaitant repr�senter un syst�me et son �volution suivant diff�rents stimulis, al�atoires ou non, il paraissait judicieux de s'orienter vers un sujet de ce type.
Ayant initialement choisi Psychologie et le projet �tant d�j� attribu� � un autre groupe, le choix d'histoire parut logique.
\subsection{Objet}
Repr�senter sous la forme d'un journal agr�ment� d'objets graphiques l'�volution d'un nombre limit� de peuples au cours du temps et en fonction d'un certain nombre d'�v�nements, al�atoires ou non.
\subsection{Organisation}
\begin{itemize}
\bulletitem Matteo Staiano : Interface Graphique
\bulletitem Mathieu Hannoun : Conception backend
\bulletitem Commun : R�flexion algorithmique, debugging, compte-rendus et �l�ments de livraison finale.
\end{itemize}
\subsection{Environnements de travail et outils utilis�s}
\begin{itemize}
\bulletitem Programmation Java : Eclipse, Intellij IDEA
\bulletitem VCS : GitHub
\bulletitem Production du rapport \LaTeX{} : TeXnicCenter
\end{itemize}
\section{Sp�cifications et fonctionnalit�s finales}
\label{sec:specifications}

\subsection{Backend}
\paragraph{Peuple}
Chaque peuple commence avec des attributs principaux bien sp�cifiques dont d�coulent des attributs secondaires.
\subparagraph{Attributs principaux\protect\footnote{D�tails des relations entre attributs dans le sch�ma en derni�re page}}
\begin{itemize}
\bulletitem Ressources
\bulletitem Population
\bulletitem Agressivit�
\bulletitem Education
\bulletitem Territoire
\end{itemize}
\subparagraph{Attributs secondaires}
\begin{itemize}
\bulletitem A
\bulletitem B
\bulletitem C
\end{itemize}
\smallbreak
Les diff�rents peuples peuvent avoir des relations avec les autres sous deux formes : la guerre et le commerce. Le d�clenchement de ces relations ne d�pend que des diff�rents attributs secondaires.
\paragraph{Guerre et commerce}
Chaque tour, des guerres et des liens commerciaux commencent, ou non, entre les peuples, en fonction de leurs caract�ristiques secondaires. Ces relations n'influent que sur les attributs principaux.
\newline
La guerre est co�teuse en population pour les deux peuples, mais apporte richesse et territoire au peuple disposant de la plus grande puissance militaire.
\newline
Le commerce apporte un b�n�fice mutuel aux deux peuples, mais plus un peuple dispose de puissance politique et plus il sera capable de tirer b�n�fice d'un commerce avec un autre.
%\section{R�alisation}
\label{sec:impl}

%\begin{figure}
%\centering
%\includegraphics[width=3.5cm, height=2cm]{images/programmer.png}
%\caption{Un programmeur occup�}
%\label{fig:modele}
%\end{figure}

\fig{images/histoire.jpg}{8cm}{7cm}{Un programmeur occup�}{programmeur}


\paragraph{} Dans la figure \ref{fig:programmeur}, on peut voir un programmeur tr�s occup� par son travail.


\begin{algorithm}
\caption{Calculate $y = x^n$}
\begin{algorithmic} 
\REQUIRE $n \geq 0 \vee x \neq 0$
\ENSURE $y = x^n$
\STATE $y \leftarrow 1$
\IF{$n < 0$}
\STATE $X \leftarrow 1 / x$
\STATE $N \leftarrow -n$
\ELSE
\STATE $X \leftarrow x$
\STATE $N \leftarrow n$
\ENDIF
\WHILE{$N \neq 0$}
\IF{$N$ is even}
\STATE $X \leftarrow X \times X$
\STATE $N \leftarrow N / 2$
\ELSE[$N$ is odd]
\STATE $y \leftarrow y \times X$
\STATE $N \leftarrow N - 1$
\ENDIF
\ENDWHILE
\end{algorithmic}
\end{algorithm}

%%% Une autre fa�on pour �crire un algorithme %%%
%\begin{algorithm}[H]
 %\KwData{this text}
 %\KwResult{how to write algorithm with \LaTeX2e }
 %initialization\;
 %\While{not at end of this document}{
  %read current\;
  %\eIf{understand}{
   %go to next section\;
   %current section becomes this one\;
   %}{
   %go back to the beginning of current section\;
  %}
 %}
 %\caption{How to write algorithms}
%\end{algorithm}
%\section{Manuel Utilisateur}
\label{sec:manuel}

\noindent Cette section est d�di�e au manuel utilisateur. 

%\section{D�roulement du projet}
\label{sec:deroulement}

\noindent Dans cette section, nous d�crivons comment la r�alisation du projet s'est d�roul�e au sein de l'�quipe de projet. La r�partition des t�ches, la synchronisation du travail et l'utilisation du temps seront abord�es. 

%\section{Conclusion}
\label{sec:conclusion}

\noindent Dans cette section, nous r�sumons la r�alisation du projet et nous pr�sentons �galement les extensions et am�liorations possibles du projet.


%R�f�rences bibliographiques du document
\bibliographystyle{plain}
\bibliography{bibliographies}
\nocite{*}

\end{document}
