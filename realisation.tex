\section{R�alisation}
\label{sec:impl}

%\begin{figure}
%\centering
%\includegraphics[width=3.5cm, height=2cm]{images/programmer.png}
%\caption{Un programmeur occup�}
%\label{fig:modele}
%\end{figure}

\fig{images/histoire.jpg}{8cm}{7cm}{Un programmeur occup�}{programmeur}


\paragraph{} Dans la figure \ref{fig:programmeur}, on peut voir un programmeur tr�s occup� par son travail.


\begin{algorithm}
\caption{Calculate $y = x^n$}
\begin{algorithmic} 
\REQUIRE $n \geq 0 \vee x \neq 0$
\ENSURE $y = x^n$
\STATE $y \leftarrow 1$
\IF{$n < 0$}
\STATE $X \leftarrow 1 / x$
\STATE $N \leftarrow -n$
\ELSE
\STATE $X \leftarrow x$
\STATE $N \leftarrow n$
\ENDIF
\WHILE{$N \neq 0$}
\IF{$N$ is even}
\STATE $X \leftarrow X \times X$
\STATE $N \leftarrow N / 2$
\ELSE[$N$ is odd]
\STATE $y \leftarrow y \times X$
\STATE $N \leftarrow N - 1$
\ENDIF
\ENDWHILE
\end{algorithmic}
\end{algorithm}

%%% Une autre fa�on pour �crire un algorithme %%%
%\begin{algorithm}[H]
 %\KwData{this text}
 %\KwResult{how to write algorithm with \LaTeX2e }
 %initialization\;
 %\While{not at end of this document}{
  %read current\;
  %\eIf{understand}{
   %go to next section\;
   %current section becomes this one\;
   %}{
   %go back to the beginning of current section\;
  %}
 %}
 %\caption{How to write algorithms}
%\end{algorithm}