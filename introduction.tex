\section{Introduction}
\label{sec:introduction}

%Il n'y a pas d'espace au d�but du paragraphe.
\noindent Ceci est une d�monstration d'un document r�dig� en utilisant LaTex.

Pour commencer un nouveau paragraphe, il suffit de sauter une ligne.

\paragraph{Fonctionnalit�s} Fonctionnalit�s du programme:
\begin{itemize}
\item Calculer une formule math�matique, par exemple, 
\item Extraire les informations utiles depuis la formule.
\item Afficher les r�sultats.
\end{itemize}

\paragraph{} Dans \cite{data}, les auteurs ont voulu montrer blabla...

\paragraph{} Dans la formule math�mtique \ref{eq:f1}, on peut voir que blabla...

\begin{equation}
\cos (2\theta) = \cos^2 \theta - \sin^2 \theta
\label{eq:f1}
\end{equation}

\paragraph{} Outils de d�veloppement:
\begin{enumerate}
\item Java
\item Eclipse
\item Latex
\end{enumerate}
